\documentclass[12pt, oneside]{article}   	% use "amsart" instead of "article" for AMSLaTeX format
\usepackage{geometry}                		% See geometry.pdf to learn the layout options. There are lots.
\geometry{letterpaper}                   		% ... or a4paper or a5paper or ... 
%\geometry{landscape}                		% Activate for rotated page geometry
\usepackage[parfill]{parskip}    		% Activate to begin paragraphs with an empty line rather than an indent
\usepackage{graphicx}				% Use pdf, png, jpg, or eps§ with pdflatex; use eps in DVI mode
								% TeX will automatically convert eps --> pdf in pdflatex	\
								
\usepackage{datetime2}	
\DTMsetdatestyle{mdyy}
\DTMsetup{datesep=/}
								
\usepackage{amssymb, amsmath}

%SetFonts
\usepackage[sc,slantedGreek]{mathpazo}

%SetFonts
\usepackage{fancyhdr}
\pagestyle{fancy}
\fancyhf{}
\rhead{PSAM Week One Solution to Lay 1.7.9 by Pyxie Star | Page \thepage}
\lhead{\today}
\renewcommand{\headrulewidth}{1pt}

\begin{document}

\paragraph*{\textbf{1.7.9 (a) For what values of $h$ is $\vec{v_3}$ in Span $\{\vec{v_1},\vec{v_2}\}$, and (b) for what values of $h$ is $\{\vec{v_1},\vec{v_2},\vec{v_3}\}$ linearly \emph{dependent}?
Justify each answer.}}
\[ \vec{v_1}= \left[ \begin{array}{r}
1\\
-3\\
2\end{array}\right ] , 
 \vec{v_2}= \left[ \begin{array}{r}
-3\\
9\\
-6\end{array}\right ] ,
\vec{v_3}= \left[ \begin{array}{r}
5\\
-7\\
h\end{array}\right ] \]

\paragraph{a)} $\vec{v_2}$ is a scalar multiple of $\vec{v_1}$. Specifically, 
\begin{equation}
\vec{v_2}=-3\vec{v_1}
\end{equation}

This means that Span $\{\vec{v_1},\vec{v_2}\}$ is the set of all vectors that lie on the line that goes through $\vec{v_1}$ (and $\vec{v_2}$) and $\vec{0}$. 

\begin{center}
\includegraphics[width=80mm]{Span.jpg}
\end{center}

All of the vectors in Span $\{\vec{v_1},\vec{v_2}\}$ must be parallel to $\vec{v_1}$ so that they lie on that line. So to have $\vec{v_3}$ in Span $\{\vec{v_1},\vec{v_2}\}$, we need
\begin{equation}\label{da}
\vec{v_3}=c_1\vec{v_1}
\end{equation}
where $c_1$ is some constant.

This gives three equalities:
\begin{align}
5&=c_1 (1)\label{1}\\
-7&=c_1(-3)\label{2}\\
h&=c_1(2)\label{3}
\end{align}
Solving Eq\eqref{1} for $c_1$ gives
\begin{equation}
c_1=5
\end{equation}
But substituting this value into Eq\eqref{2} gives
\begin{equation}
-3(5)=-7
\end{equation}
which is simply not true.

This can also be shown by turning Eq\eqref{da} into the augmented matrix $[\vec{v_1}  \,\,\vec{v_3}]$ and looking for solutions.

\[  \left[ \begin{array}{rr}
1&5\\
-3&-7\\
2&h\end{array}\right ]\rightarrow
\left[ \begin{array}{rr}
1&5\\
0&8\\
2&h\end{array}\right ]\]

\begin{equation}
0\neq8
\end{equation}

The system is inconsistent regardless of the value of $h$, so there are $\boxed{\textrm{no values of } h}$ that make $\vec{v_3}$ in Span $\{\vec{v_1},\vec{v_2}\}$.

\paragraph{b)} $\{\vec{v_1},\vec{v_2},\vec{v_3}\}$ is linearly dependent if the equation
\begin{equation}\label{6}
c_1\vec{v_1}+c_2\vec{v_2}+c_3\vec{v_3}=\vec{0}
\end{equation}
 does not have only the trivial solution. That is, there must be $c_1, \,c_2,\,c_3$ not all zero that satisfy Eq\eqref{6}. 

As stated previously, we know that  
\begin{equation}
\vec{v_2}=-3\vec{v_1}
\end{equation}

Choosing 
\begin{align}
c_1&=-3\\
c_2&=-1\\
c_3&=0
\end{align}

and substituting into Eq\eqref{6} gives
\begin{equation}
\begin{split}
-3\vec{v_1}-\vec{v_2}+0(\vec{v_3})&=-3\vec{v_1}-\vec{v_2}\\
&=\vec{v_2}-\vec{v_2}\\
&=\vec{0}
\end{split}
\end{equation}

Thus there does exist $c_1, \,c_2,\,c_3$ not all zero that satisfy Eq\eqref{6}. This is true regardless of the value of $h$.

This can also be shown by checking the matrix$[\vec{v_1}\,\,\vec{v_2}\,\,\vec{v_3}\,\,\vec{0}]$ for solutions other than the trivial solution.

\[  \left[ \begin{array}{rrrr}
1&-3&5&0\\
-3&9&-7&0\\
2&-6&h&0\end{array}\right ]\rightarrow
\left[ \begin{array}{rrrr}
1&-3&5&0\\
0&0&8&0\\
2&-6&h&0\end{array}\right ]\rightarrow
\left[ \begin{array}{rrcr}
1&-3&5&0\\
0&0&8&0\\
0&0&h-10&0\end{array}\right ]\]

At this point we can see that the system is consistent (because there is no pivot position in the rightmost column) and that $x_2$ is free (because there is no pivot position in the second column), so there are infinitely many solutions regardless of the value of $h$.


So $\{\vec{v_1},\vec{v_2},\vec{v_3}\}$ is linearly dependent for $\boxed{\textrm{all values of }h}$.

\end{document}  